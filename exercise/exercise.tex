\documentclass{jsarticle}
\usepackage{bm}
\usepackage{amsmath}

\renewcommand{\labelenumi}{(\arabic{enumi})}

\begin{document}

\begin{center}
    {\huge 数理物理 予想問題集}\\
\end{center}

【1】3次元空間に2つの質点A, Bがある。これらの質点の座標をそれぞれ$\bm{r}_A = (x_A,y_A,z_A), \bm{r}_B = (x_B,y_B,z_B)$とする時、これらが相対距離$r$に依存するポテンシャル$U(r)$により相互作用をしている。ただし$r$は以下のように定義する。

$$
    \begin{aligned}
        r & = \sqrt{(\boldsymbol{r}_A -\boldsymbol{r}_B})^2 \\
          & = \sqrt{(x_A-x_B)^2+(y_A-y_B)^2+(z_A-z_B)^2}
    \end{aligned}
$$

質点の質量をどちらも$m$とする時、以下の問いに答えよ。

\begin{enumerate}
    \item 2つの物体に働く力が、作用・反作用の法則を満たすことを示せ($x$成分のみ示せば良い)。
    \item この系の$x$方向の重心の運動量$p_x = m(\dot{x}_1 + \dot{x}_1)$が運動により保存することを示せ。

\end{enumerate}



\end{document}