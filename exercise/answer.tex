\documentclass{jsarticle}
\usepackage{bm}
\usepackage{amsmath}
\usepackage[dvipdfmx]{graphics}
\usepackage{here}
\usepackage[margin=20truemm]{geometry}
\usepackage{enumitem}
\usepackage{cancel}
\renewcommand{\labelenumi}{【\arabic{enumi}】}
\renewcommand{\labelenumii}{(\arabic{enumii})}
\setlist[enumerate]{itemsep=10pt}


\begin{document}

\begin{center}
  {\huge 数理物理中間試験 予想問題集 解答例}\\
\end{center}

\begin{enumerate}
  \item
        \begin{enumerate}
          \item $$
                  r^2 = (x_A-x_B)^2+(y_A-y_B)^2+(z_A-z_B)^2
                $$
                であることから、
                $$
                  \frac{\partial r}{\partial x_A} = \frac{(x_A-x_B)}{r}
                $$
                同様に、$r$の$x_B$による偏微分は
                $$
                  \frac{\partial r}{\partial x_B} = -\frac{(x_A-x_B)}{r}
                $$
                以上を用いると、物体Aの$x$成分の運動方程式は
                $$
                  \begin{aligned}
                    m \ddot{x}_A & = - \frac{\partial U(r)}{\partial x_A}                \\
                                 & =- \frac{d U(r)}{d r} \frac{\partial r}{\partial x_A} \\
                                 & = -U' \frac{(x_A-x_B)}{r}
                  \end{aligned}
                $$

                同様に、物体Bに働く$x$成分の運動方程式は
                $$
                  m \ddot{x}_B = U'\frac{(x_A-x_B)}{r}
                $$

                以上から、物体AB間に働く力は、大きさは同じで向きが逆である、すなわち作用・反作用の法則が成り立つことがわかる。
          \item 物体A,Bの運動方程式は

                $$
                  \begin{aligned}
                    m\ddot{x}_A & = -U' \frac{x_A - x_B}{r} \\
                    m\ddot{x}_B & = U' \frac{x_A - x_B}{r}  \\
                  \end{aligned}
                $$
                両辺の和を取ると、
                $$
                  m (\ddot{x}_A + \ddot{x}_B) = 0
                $$
                ここで、$p_x$の時間微分を計算すると、
                $$
                  \begin{aligned}
                    \dot{p}_x & = m (\ddot{x}_A+\ddot{x}_B) \\
                              & = 0
                  \end{aligned}
                $$
                以上から、$p_x$は保存量となる。
        \end{enumerate}
  \item
        $$
          r = \sqrt{x^2+y^2}
        $$
        であるから、
        $$
          \begin{aligned}
            \frac{\partial r}{\partial x} & = \frac{x}{r} \\
            \frac{\partial r}{\partial y} & = \frac{y}{r}
          \end{aligned}
        $$
        これを用いると、運動方程式は
        $$
          \begin{aligned}
            m \ddot{x} & = - \partial_x U(r) = -U'(r) \frac{\partial r}{\partial x}= - U'(r) \frac{x}{r} \\
            m \ddot{y} & = - \partial_y U(r) = -U'(r) \frac{\partial r}{\partial y}= - U'(r) \frac{y}{r}
          \end{aligned}
        $$
        従って、
        $$
          m \dot{x} y - m\ddot{y} x = 0
        $$
        角運動量の時間微分$\dot{L}$は
        $$
          \begin{aligned}
            \dot{L} & = m \frac{d}{dt} (\dot{x} y - \dot{y} x)                                         \\
                    & = m (\ddot{x} y + \cancel{\dot{x}\dot{y}} - \ddot{y}x - \cancel{\dot{y}\dot{x}}) \\
                    & = m \ddot{x} y - m \ddot{y} x                                                    \\
                    & =0
          \end{aligned}
        $$
        以上から、角運動量$L$が保存量であることが示された。
\end{enumerate}


\end{document}