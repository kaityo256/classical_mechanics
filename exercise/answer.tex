\documentclass{jsarticle}
\usepackage{bm}
\usepackage{amsmath}
\usepackage[dvipdfmx]{graphics}
\usepackage{here}
\usepackage[margin=20truemm]{geometry}
\usepackage{enumitem}
\usepackage{cancel}
\renewcommand{\labelenumi}{【\arabic{enumi}】}
\renewcommand{\labelenumii}{(\arabic{enumii})}
\setlist[enumerate]{itemsep=10pt}


\begin{document}

\begin{center}
  {\huge 数理物理中間試験 予想問題集 解答例}\\
\end{center}

\begin{enumerate}
  \item
        \begin{enumerate}
          \item $$
                  r^2 = (x_A-x_B)^2+(y_A-y_B)^2+(z_A-z_B)^2
                $$
                であることから、
                $$
                  \frac{\partial r}{\partial x_A} = \frac{(x_A-x_B)}{r}
                $$
                同様に、$r$の$x_B$による偏微分は
                $$
                  \frac{\partial r}{\partial x_B} = -\frac{(x_A-x_B)}{r}
                $$
                以上を用いると、物体Aの$x$成分の運動方程式は
                $$
                  \begin{aligned}
                    m \ddot{x}_A & = - \frac{\partial U(r)}{\partial x_A}                \\
                                 & =- \frac{d U(r)}{d r} \frac{\partial r}{\partial x_A} \\
                                 & = -U' \frac{(x_A-x_B)}{r}
                  \end{aligned}
                $$

                同様に、物体Bに働く$x$成分の運動方程式は
                $$
                  m \ddot{x}_B = U'\frac{(x_A-x_B)}{r}
                $$

                以上から、物体AB間に働く力は、大きさは同じで向きが逆である、すなわち作用・反作用の法則が成り立つことがわかる。
          \item $p_x$の時間微分を計算すると、
                $$
                  \dot{p}_x = m(\ddot{x}_A + \ddot{x}_B)
                $$
                ここで、物体A,Bの運動方程式は

                $$
                  \begin{aligned}
                    m\ddot{x}_A & = -U' \frac{x_A - x_B}{r} \\
                    m\ddot{x}_B & = U' \frac{x_A - x_B}{r}  \\
                  \end{aligned}
                $$
                であるから、両辺の和を取ると、
                $$
                  m (\ddot{x}_A + \ddot{x}_B) = 0
                $$
                以上から、$\dot{p}_x=0$となるため、$p_x$は保存量となる。
        \end{enumerate}
  \item 角運動量の時間微分は
        $$
          \begin{aligned}
            \dot{L} & = m \frac{d}{dt} (\dot{x} y - \dot{y} x)                                         \\
                    & = m (\ddot{x} y + \cancel{\dot{x}\dot{y}} - \ddot{y}x - \cancel{\dot{y}\dot{x}}) \\
                    & = m \ddot{x} y - m \ddot{y} x                                                    \\
          \end{aligned}
        $$
        また、
        $$
          r = \sqrt{x^2+y^2}
        $$
        であるから、
        $$
          \begin{aligned}
            \frac{\partial r}{\partial x} & = \frac{x}{r} \\
            \frac{\partial r}{\partial y} & = \frac{y}{r}
          \end{aligned}
        $$
        これを用いると、運動方程式は
        $$
          \begin{aligned}
            m \ddot{x} & = - \partial_x U(r) = -U'(r) \frac{\partial r}{\partial x}= - U'(r) \frac{x}{r} \\
            m \ddot{y} & = - \partial_y U(r) = -U'(r) \frac{\partial r}{\partial y}= - U'(r) \frac{y}{r}
          \end{aligned}
        $$
        従って、
        $$
          m \dot{x} y - m\ddot{y} x = 0
        $$
        以上から、$\dot{L}=0$であるから、$L$が保存量であることが示された。
  \item 質量$m$の物体を$\delta l$だけ仮想的に下に動かすことを考える。物体$m$には下方向に重力による力$mg$がかかっているため、$m$が受ける仕事は$-mg \delta l$となる。

        一方、斜面の上の物体は$\delta l$だけ斜面を登る。物体$M$には斜面を下る方向に$Mg \sin \theta$の力が加わっており、それに逆らう方向に動くから、$M$が受ける仕事は$Mg \sin \theta \delta l$である。

        仮想仕事の原理より、釣り合いの位置からの仮想変位による仕事はゼロであるから、

        $$
          -mg \delta l + Mg \sin \theta \delta l = 0
        $$

        以上から、

        $$
          m = M\sin \theta
        $$
  \item
        $$
          \begin{aligned}
            \frac{dB}{dx} & = \frac{d}{dx} \left( F - f' \frac{\partial F}{\partial f'}\right)                  \\
                          & = \frac{dF}{dx} - f'' \frac{\partial F}{\partial f'}
            -  f' \underbrace{\frac{d}{dx}\left(\frac{\partial F}{\partial f'}\right)}_{=\partial L/\partial f} \\
                          & = \frac{\partial F}{\partial f}f' + \frac{\partial F}{\partial f'}f''
            - f'' \frac{\partial F}{\partial f'} - f' \frac{\partial F}{\partial f}                             \\
                          & = 0
          \end{aligned}
        $$

        以上から、$dB/dx = 0$が示された。
  \item
        \begin{enumerate}
          \item 速度が$l\dot{\theta}$であるから、運動エネルギーは以下で与えられる。

                $$
                  K= \frac{1}{2}m l^2 \dot{\theta}^2
                $$
          \item 角度$\theta$の時、$\theta=0$の時に比べて高さが$l(1-\cos\theta)$だけ高くなるため、ポテンシャルエネルギーは以下で与えられる。

                $$
                  U = mgl (1- \cos \theta)
                $$

          \item ラグランジアンは$L=K-U$で与えられるため、この系のラグランジアンは以下で与えられる。
                $$
                  L(\theta, \dot{\theta}) = \frac{1}{2}m l^2 \dot{\theta}^2 - mlg (1- \cos \theta)
                $$
                ここからオイラー・ラグランジュ方程式を求めると、
                $$
                  \begin{aligned}
                    \frac{\partial L}{\partial \dot{\theta}}                          & = ml^2 \dot{\theta}  \\
                    \frac{d}{dt}\left(\frac{\partial L}{\partial \dot{\theta}}\right) & = ml^2 \ddot{\theta} \\
                    \frac{\partial L}{\partial \theta}                                & = - mlg\sin \theta
                  \end{aligned}
                $$
                以上から、オイラー・ラグランジュ方程式は
                $$
                  \begin{aligned}
                    \frac{d}{dt}\left(\frac{\partial L}{\partial \dot{\theta}}\right)  - \frac{\partial L}{\partial \theta}
                     & = ml^2 \ddot{\theta} + mlg\sin \theta \\
                     & = 0
                  \end{aligned}
                $$
                これを整理すると、以下の運動方程式を得る。
                $$
                  \ddot{\theta} = -\frac{g}{l} \sin \theta
                $$
        \end{enumerate}
  \item
        \begin{enumerate}
          \item 極座標表示を時間微分する
                $$
                  \begin{aligned}
                    \dot{x} & = \dot{r}\cos \theta - r \dot{\theta} \sin \theta \\
                    \dot{y} & = \dot{r}\sin \theta + r \dot{\theta} \cos \theta
                  \end{aligned}
                $$
                この系の運動エネルギーは$m(\dot{x}^2 + \dot{y}^2)/2$であることから、
                $$
                  K = \frac{1}{2}m(\dot{r}^2 + r^2 \dot{\theta}^2)
                $$
          \item $$
                  \begin{aligned}
                    L & = K - U                                               \\
                      & = \frac{1}{2}m(\dot{r}^2 + r^2 \dot{\theta}^2) - U(r)
                  \end{aligned}
                $$
          \item $r$に関するオイラー・ラグランジュ方程式は、

                $$
                  \begin{aligned}
                    \frac{d}{dt}\left(\frac{\partial L}{\partial \dot{r}}\right) - \frac{\partial L}{\partial r}
                     & = m \ddot{r} - mr\dot{\theta}^2 + \frac{\partial U}{\partial r} \\
                     & = 0
                  \end{aligned}
                $$

                整理すると、

                $$
                  m(\ddot{r} - r \dot{\theta}^2)= -\frac{\partial U}{\partial r}
                $$

                同様に、$\theta$に関するオイラー・ラグランジュ方程式は、

                $$
                  \begin{aligned}
                    \frac{d}{dt}\left(\frac{\partial L}{\partial \dot{\theta}}\right) - \frac{\partial L}{\partial \theta} & = \frac{d}{dt}(r^2\dot{\theta}) \\
                                                                                                                           & = 0
                  \end{aligned}
                $$

                以上をまとめると、運動方程式は

                $$
                  \begin{aligned}
                    m(\ddot{r} - r \dot{\theta}^2) & = -\frac{\partial U}{\partial r} \\
                    \frac{d}{dt}(r^2\dot{\theta})  & = 0
                  \end{aligned}
                $$


        \end{enumerate}
\end{enumerate}


\end{document}