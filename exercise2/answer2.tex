\documentclass{jsarticle}
\usepackage{bm}
\usepackage{amsmath}
\usepackage[dvipdfmx]{graphics}
\usepackage{here}
\usepackage[margin=20truemm]{geometry}
\usepackage{enumitem}
\usepackage{cancel}
\renewcommand{\labelenumi}{【\arabic{enumi}】}
\renewcommand{\labelenumii}{(\arabic{enumii})}
\setlist[enumerate]{itemsep=10pt}


\begin{document}

\begin{center}
  {\huge 数理物理講義後半 予想問題解答例}\\
\end{center}

\begin{enumerate}
  \item
        \begin{enumerate}
          \item $p_1 = \partial L/\partial \dot{q}_1 = m\dot{q}_1$,$p_2 = \partial L/\partial \dot{q}_2 = m\dot{q}_2$より、
                $$
                  H = \frac{1}{2m}(p_1^2+p_2^2) + U(q_1-q_2)
                $$
          \item ラグランジアンを$Q_1, Q_2$で表すと、
                $$
                  L = \frac{m}{4}(\dot{Q}_1^2+\dot{Q}_2^2) - U(Q_2)
                $$
                したがって、
                $$
                  \begin{aligned}
                    P_1 & = \frac{\partial L}{\partial \dot{Q}_1} = \frac{m \dot{Q}_1}{2} \\
                    P_2 & = \frac{\partial L}{\partial \dot{Q}_2} = \frac{m \dot{Q}_2}{2} \\
                  \end{aligned}
                $$
          \item
                $$
                  H = P_1 \dot{Q}_1 + P_1 \dot{Q}_1 - L
                $$
                であるから、
                $$
                  H = \frac{1}{m}(P_1^2+P_2^2) + U(Q_2)
                $$
          \item
                $$
                  \begin{aligned}
                    \{P_1, H\} & = \frac{\partial P_1}{\partial Q_1}\frac{\partial H}{\partial P_1}
                    +\frac{\partial P_1}{\partial Q_2}\frac{\partial H}{\partial P_2}
                    - \frac{\partial P_1}{\partial P_1}\frac{\partial H}{\partial Q_1}
                    - \frac{\partial P_1}{\partial P_2}\frac{\partial H}{\partial Q_2}              \\
                               & = - \frac{\partial H}{\partial Q_1}                                \\
                               & = 0
                  \end{aligned}
                $$
          \item
                $$
                  \begin{aligned}
                    \{P_2, H\} & = \frac{\partial P_2}{\partial Q_1}\frac{\partial H}{\partial P_1}
                    +\frac{\partial P_2}{\partial Q_2}\frac{\partial H}{\partial P_2}
                    - \frac{\partial P_2}{\partial P_1}\frac{\partial H}{\partial Q_1}
                    - \frac{\partial P_2}{\partial P_2}\frac{\partial H}{\partial Q_2}              \\
                               & = - \frac{\partial H}{\partial Q_2}                                \\
                               & = -U'(Q_2)
                  \end{aligned}
                $$
        \end{enumerate}
  \item
        \begin{enumerate}
          \item
                $$
                  \begin{aligned}
                    \{L, H\} & =
                    \frac{\partial L}{\partial x}\frac{\partial H}{\partial p_x}
                    -\frac{\partial L}{\partial p_x}\frac{\partial H}{\partial x}
                    +\frac{\partial L}{\partial y}\frac{\partial H}{\partial p_y}
                    -\frac{\partial L}{\partial p_y}\frac{\partial H}{\partial y}         \\
                             & = p_y \times \frac{p_x }{m} - (-y) \times U'(r)\frac{x}{r}
                    - p_x \times \frac{p_y }{m} - x \times  U'(r)\frac{y}{r}              \\
                             & = 0
                  \end{aligned}
                $$
          \item $R(\varepsilon)$を$\varepsilon$について一次まで展開すると
                $$
                  \begin{aligned}
                    R(\varepsilon) & = \begin{pmatrix}
                                         \cos \varepsilon & -\sin \varepsilon \\
                                         \sin \varepsilon & \cos \varepsilon
                                       \end{pmatrix}             \\
                                   & = \begin{pmatrix}
                                         1           & - \varepsilon \\
                                         \varepsilon & 1
                                       \end{pmatrix} + O(\varepsilon^2)                 \\
                                   & = I - \varepsilon \begin{pmatrix}
                                                         0 & - 1 \\
                                                         1 & 0
                                                       \end{pmatrix} + O(\varepsilon^2)
                  \end{aligned}
                $$
                以上から、
                $$
                  G =
                  \begin{pmatrix}
                    0 & -1 \\
                    1 & 0
                  \end{pmatrix}
                $$
          \item
                $$
                  \begin{aligned}
                    X   & = x - \varepsilon y      \\
                    Y   & = \varepsilon x  + y     \\
                    P_x & = p_x - \varepsilon p_y  \\
                    P_y & = \varepsilon p_x  + p_y \\
                  \end{aligned}
                $$
                であるので、
                $$
                  \begin{aligned}
                    \{X, P_x\} & = \frac{\partial X}{\partial x}\frac{\partial P_x}{\partial p_x}
                    - \frac{\partial X}{\partial p_x}\frac{\partial P_x}{\partial x}
                    +\frac{\partial X}{\partial y}\frac{\partial P_x}{\partial p_y}
                    - \frac{\partial X}{\partial p_y}\frac{\partial P_x}{\partial y}              \\
                               & = 1 + \varepsilon^2
                  \end{aligned}
                $$
          \item
                $$
                  \begin{aligned}
                    \{X, P_y\} & = \frac{\partial X}{\partial x}\frac{\partial P_y}{\partial p_x}
                    - \frac{\partial X}{\partial p_x}\frac{\partial P_y}{\partial x}
                    +\frac{\partial X}{\partial y}\frac{\partial P_y}{\partial p_y}
                    - \frac{\partial X}{\partial p_y}\frac{\partial P_y}{\partial y}              \\
                               & = \varepsilon - \varepsilon                                      \\
                               & =0
                  \end{aligned}
                $$
        \end{enumerate}
  \item
        \begin{enumerate}
          \item
                $$
                  dx = \cos \theta dr - r \sin \theta d \theta
                $$
          \item
                $$
                  dy = \sin \theta dr + r \cos \theta d \theta
                $$
          \item
                $$
                  \begin{aligned}
                    dx \wedge dy & = (\cos \theta dr - r \sin \theta d \theta)\wedge (\sin \theta dr + r \cos \theta d \theta) \\
                                 & = \cos \theta \sin \theta \underbrace{dr \wedge dr}_{=0}
                    + r \cos^2 \theta dr \wedge d\theta
                    - r \sin^2 \theta \underbrace{d\theta\wedge dr}_{=-dr \wedge d\theta}
                    - r^2 \sin \theta \cos \theta \underbrace{d \theta \wedge \theta}_{=0}                                     \\
                                 & = r^2 (\cos^2 \theta + \sin^2 \theta) dr \wedge d \theta                                    \\
                                 & = r dr \wedge d \theta
                  \end{aligned}
                $$
        \end{enumerate}
  \item
        \begin{enumerate}
          \item
                $$
                  \begin{aligned}
                    x        & = -\frac{\partial W_3}{\partial p_x} = r \cos \theta                             \\
                    y        & = -\frac{\partial W_3}{\partial p_y} = r \sin \theta                             \\
                    p_r      & = -\frac{\partial W_3}{\partial r} = p_x \cos \theta + p_y \sin \theta           \\
                    p_\theta & = -\frac{\partial W_3}{\partial \theta} = -p_x r \sin \theta + p_y r \cos \theta
                  \end{aligned}
                $$
          \item
                $$
                  \begin{aligned}
                    p_r                & = p_x \cos \theta + p_y \sin \theta  \\
                    \frac{p_\theta}{r} & = -p_x \sin \theta + p_y \cos \theta
                  \end{aligned}
                $$
                であるから、両辺に$\cos \theta$や$\sin \theta$をかけて和や差をとると
                $$
                  \begin{aligned}
                    p_x & = p_r \cos \theta - \frac{p_\theta}{r} \sin \theta \\
                    p_y & = p_r \sin \theta + \frac{p_\theta}{r} \cos \theta \\
                  \end{aligned}
                $$
          \item
                $$
                  \begin{aligned}
                    \{x, p_x\} & = \frac{\partial x}{\partial r}\frac{\partial p_x}{\partial p_r}
                    -
                    \underbrace{\frac{\partial x}{\partial p_r}}_{=0}\frac{\partial p_x}{\partial r}
                    + \frac{\partial x}{\partial \theta}\frac{\partial p_x}{\partial p_\theta}
                    -
                    \underbrace{\frac{\partial x}{\partial p_\theta}}_{=0}\frac{\partial p_x}{\partial \theta} \\
                               & = \cos^2 \theta + \sin^2 \theta                                               \\
                               & = 1
                  \end{aligned}
                $$
        \end{enumerate}
\end{enumerate}

\end{document}