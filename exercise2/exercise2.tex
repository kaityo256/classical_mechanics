\documentclass{jsarticle}
\usepackage{bm}
\usepackage{amsmath}
\usepackage[dvipdfmx]{graphics}
\usepackage{here}
\usepackage[margin=20truemm]{geometry}
\usepackage{enumitem}
\usepackage{cancel}
\renewcommand{\labelenumi}{【\arabic{enumi}】}
\renewcommand{\labelenumii}{(\arabic{enumii})}
\setlist[enumerate]{itemsep=10pt}


\begin{document}

\begin{center}
  {\huge 数理物理期末試験 予想問題集}\\
\end{center}

\begin{enumerate}
  \item 以下の1次元2粒子系の以下のラグランジアンを考える。ただし粒子の質量はいずれも$m$である。
        $$
          L = \frac{m}{2}(\dot{q}_1^2+\dot{q}_2^2) - U(q_1-q_2)
        $$
        この時、以下の問に答えよ。
        \begin{enumerate}
          \item 座標$q_1, q_2$について正準共役な運動量$p_1, p_2$とする時、この系のハミルトニアン$H$を$q_1, q_2, p_1, p_2$を用いて表わせ。
          \item $Q_1 = q_1 + q_2, Q_2 = q_1 - q_2$という変数変換を行った時、$Q_1, Q_2$に正準共役な運動量$P_1, P_2$をそれぞれ求めよ。
          \item 変換後の変数$Q_1, Q_2, P_1, P_2$でハミルトニアンを表わせ。
          \item ポアソン括弧$\{P_1, H\}$を求めよ。
          \item ポアソン括弧$\{P_2, H\}$を求めよ。
        \end{enumerate}

\end{enumerate}


\end{document}